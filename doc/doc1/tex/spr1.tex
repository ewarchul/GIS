\documentclass{article}
\usepackage[utf8]{inputenc}
\usepackage{amssymb}
\usepackage{cite}
\usepackage{listings}
\usepackage{color}
\usepackage{amsmath}
\setcounter{tocdepth}{3}
\usepackage{graphicx}
\usepackage{polski}
\usepackage{amssymb}
\newcommand*{\QEDA}{\hfill\ensuremath{\blacksquare}}%
\newcommand*{\QEDB}{\hfill\ensuremath{\square}}%<Paste>
\usepackage[a4paper]{geometry}
\usepackage{psfrag}
\usepackage{bbm}
\usepackage[T1]{fontenc}
\usepackage{color}
\usepackage{url}
\usepackage[dvipsnames]{xcolor}
\usepackage{hyperref}
\hypersetup{
    colorlinks=true,
    linkcolor=violet,
    filecolor=magenta,      
    citecolor=blue,
    urlcolor=cyan,
}
\usepackage{graphicx}
\graphicspath{{graphics/}}
\usepackage{float}
\usepackage{mathtools}
\DeclarePairedDelimiter\ceil{\lceil}{\rceil}
\DeclareMathOperator*{\argmin}{argmin}
\DeclarePairedDelimiter\floor{\lfloor}{\rfloor}

\DeclareGraphicsExtensions{.eps}
\DeclareGraphicsExtensions{.ps}
\usepackage{algorithmic}
\usepackage{psfrag}

\newcommand{\wek}[1]{
	{\bf{#1}} 
}
\newcommand{\jed}[1]{
	{$\left[#1\right]$}
}
\newcommand{\mat}[1]{
	{\bf #1} 
}
\newcommand{\todo}[1]{
	\colorbox{yellow} {{\color{red}
	\emph {TODO: #1}
}}}
\newcommand{\srednia}[1]{
	\langle #1 \rangle 
}
\newcommand{\argmax}{\operatornamewithlimits{argmax}} 
\title{Grafy i sieci: \\ Generator sieci bezskalowej II (model Barabasiego-Albert)}
%lista i kolejnosc na niej do ustalenia
\author{Eryk Warchulski \\ Kanstantsin Padmostka \\ Prowadzący: dr inż. Sebastian Kozłowski}%
\date{\today\\wer. 1.0}
\begin{document}
\maketitle
{\footnotesize{\tableofcontents}}
\vspace{2cm}
\begin{abstract}
	Dokument ten zawiera szczegółowy opis zadania projektowego, który ma potwierdzić zrozumienie tematu przez autorów.
	Tematem zadania jest implementacja szybkiego i przenośnego generatora
	sieci bezskalowych w wybranym języku programowania. 
	W sekcji (\ref{s1}) znajduje się omówienie tematu zadania, a w sekcjach (\ref{s2}) i  
	(\ref{s3}) ogólny opis grafów losowych oraz opis modelu Barabasiego-Albert. 
\end{abstract}
\newpage
\section{Opis zadania \label{s1}}
	Postawionym przed nami zadaniem jest zaimplementowanie generatora grafów losowych w ujęciu Barabasieg-Albert, które zostanie szczegółowo
	opisane w dalszej części dokumentu (\ref{s3}). Generator poza spełnianiem swojej podstawowej funkcji musi charakteryzować się
	przenośnością oraz jak najmniejszą złożnością obliczeniową i pamięciową. Dla poprawnie działającego generatora kolejnym krokiem w realizacji
	zadania jest zbadanie rozkładów stopni wierzchołków grafów i porównanie ich z modelami teoretycznymi. Pełna realizacja zadania zakłada istnienie
	możliwości zapisu wygenerowanego grafu do ustalonego formatu, co pozwoli odwtorzyć sam graf oraz przebieg eksperymentów numerycznych.\\
	Dokumentacja ta jest wolna od opisu implementacji generatora, eksperymentów numerycznych oraz sposobu ich wizualizacji. Kwestie te zostaną
	omówione w kolejnych wersjach dokumentacjach, tj. odpowiednio: wersji $2.0$ oraz $3.0$. 
\section{Grafy losowe \label{s2}}
	Stosowanie teorii grafów do modelowania zjawisk zachodzących w realnym świecie jest oparte w dużej mierze na grafach losowych. Podyktowane 
	jest to faktem, że zjawiska te i towarzyszące im zdarzenia wykazują w skali makroskopowej charakter stochastyczny. Przykłady dziedzin, w których stosowane
	są grafy losowe do modelowania pewnych zjawisk są następujące:
		\begin{itemize}
			\item 	sieci połączeń handlowych
			\item	sieci \texttt{WWW}
			\item	sieci neuronowe (rekurencyjne)
			\item	sieci społecznościowe (np. \texttt{Facebook})
		\end{itemize}
	Zdefiniowanie grafu losowego wymaga z kolei zdefiniowania struktur jak \textit{przestrzeń grafów losowych} $\mathcal{G}$, która jest
	wyposażona w unormowaną miarę $\mathbb{P}(\bullet)$ mówiącą o prawdopodobieństwie wylosowania grafu $G$ o pewnych właściwościach \cite{Fronczak1}.
	\newline
	Zadanie to ze względu na złożoną strukturę obiektów jakimi są grafy nie jest tak intuicyjne jak określenie przestrzeni probabilistycznej dla
	zdarzeń, które można reprezentować liczbami. Z tego względu istnieje szereg alternatywnych modeli, które podejmują się rozwiązania tego zadania.
	Pokrótce zostanie omówiony najstarszy i najprostszy model wprowadzony przez Erdős'a i Rényi'e jeszcze w latach 60. ubiegłego wieku \cite{Erdos1}. 
\subsection{Model E-R}
	Model ten jest oparty o dwójkę parametrów $(n, p)$: parametr $n \in \mathbb{N}$ oznacza liczbę wierzchołków generowanego grafu $G$, a $p \in
	(0, 1)$ stanowi o prawdopodobieństwie zdarzenia polegającego na zaistnieniu krawędz między każdą parą z $n^2$ wierzchołków grafu $G$. \\
	Na podstawie powyższego łatwo widać, że rozkład stopni wierzchołków w grafie zadany jest przez rozkład dwumianowy z funkcją
	gęstości prawdopodobieństwa 
	\begin{equation}
		p(n,k; p) = \binom{n-1}{k}p^{k}(1-p)^{n-k - 1}
	\end{equation}
	implikuje to fakt, że średni stopień wrzechołka $\mathbb{E}deg(v)$ wynosi $(n-1)p$. Ponadto, prawdopodobieństwo wylosowania grafu E-R o $e$ krawędziach i $n$ wierzchołkach wynosi
	$\binom{\binom{n}{2}}{m}p^{m}(1-p)^{\binom{n}{2} - m}$. Na tej podstawie liczba wszystkich możliwych grafów E-R o $n$ wierzchołkach wynosi
	\begin{equation}
		\sum_{e = 0}^{\binom{n}{2}} \binom{\binom{n}{2}}{e} = 2^{\binom{n}{2}}
	\end{equation}
	przy czym $\binom{\binom{n}{2}}{e}$ oznacza liczbę $e$-elementowych kombinacji zbioru utworzonego ze wszystkich par zbioru $n$-elementowego. \newline
	Niestety, model taki nie jest najlepszym kandydatem do \textit{naśladowania} obiektów rzeczywistych. Przy $p << 1$ rozkład stopni wierzchołków dany jest rozkładem Poissona, tj. rozkładem, który stosowany jest do zdarzeń rzadkich występujących w określonym przedziale czasu. Grafy generowane w tym modelu nie są w stanie dobrze odwzorowywać \textit{hub}-ów, tj. skupisk. \newline
	Modele, które są wolne powyżej opisanych wad grafów opartych o model E-R, oparte są o rozkłady potęgowe i zostaną opisane w następnej sekcji (\ref{s3}).
\section{Model Barabasiego-Albert \label{s3}}
	\todo{nawiązanie do poprzedniego rozdziału i napisanie motywacji modeli BA w kontekście zasady maksymalnej entropii}
\subsection{Sieć bezskalowa}
	\todo{zdefiniowanie sieci bezskalowej}
\subsection{Preferencyjne dołączanie wierzchołków}
	\todo{opisanie na czym polega ten mechanizm}
\subsection{Rozkład stopni wierzchołków}
	\todo{wyprowadzenie zależności na rozkład stopni wierzchołków i napisanie, że są różne}
\subsubsection{Model czasu ciągłego}
	\todo{wyprowadzenie r.s.w. dla modelu czasu ciągłego}
\subsubsection{Model równania \textit{master}}
	\todo{to samo co wyżej}
\bibliography{spr1}{}
\bibliographystyle{plain}
\end{document}
